\cdbalgorithm{range}{}

Replaces a two-element list of integers
  with the elements in the range,
\begin{screen}{1,2,4}
{-3,2};
@range(%);
\{ -3,-2,-1,0,1,2 \}
@range[{1,3}];
\{ 1,2,3 \}
\end{screen}
When the list has three elements, the algorithm generates a list
consisting of a number of copies of the third element of the given
list,
\begin{screen}{1}
@range[{1,3,c}];
{c,c,c};
\end{screen}
When the list contains four elements, the first one is used as a
counter, and can appear in the fourth element to generate different
list elements,
\begin{screen}{1}
@range[{i, 1, 5, c_{i} }];
{ c_1, c_2, c_3, c_4, c_5 };
\end{screen}
~

\cdbseealgo{inner}
\cdbseealgo{length}
\cdbseealgo{coefficients}
