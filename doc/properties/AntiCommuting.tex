\cdbproperty{AntiCommuting}{}

Makes components anti-commuting. Example:
\begin{screen}{1,2}
{A,B}::AntiCommuting.
B A;
@prodsort!(%);
(-1) A B;
\end{screen}
It also works for objects with indices:
\begin{screen}{1,2}
{\psi_{m}, \chi}::AntiCommuting.
\psi_{m} \chi \psi_{n};
@prodsort!(%);
(-1) \chi \psi_{m} \psi_{n};
\end{screen}
If you want a pattern like \verb|\psi_{m}| to anti-commute with
itself, you should use the \subsprop{SelfAntiCommuting} property instead.

Note that this property always refers to components; if you attach it
to an object with \subsprop{ImplicitIndex} property, the commutation
property does not refer to the object as a whole, but rather to its
components. The logic behind that becomes clear when considering
e.g.~spinor bilinears
\begin{screen}{1,2,3,4,5,6,7,9}
{\chi, \psi}::Spinor(dimension=10, type=MajoranaWeyl).
{\chi, \psi}::AntiCommuting.
\bar{#}::DiracBar.
\Gamma{#}::GammaMatrix.
{\chi, \psi}::SortOrder.
\bar{\psi} \Gamma_{m n p} \chi;
@prodsort!(%);
@prodsort: not applicable.
@spinorsort!(%);
\bar{\chi} \Gamma_{m n p} \psi;
\end{screen}
Here \subscommand{prodsort} did not act because both the spinors and
the gamma matrices have the \subsprop{ImplicitIndex} property and
there are thus no simple rules for their re-ordering.

\cdbseealgo{prodsort}
\cdbseealgo{spinorsort}
\cdbseeprop{SelfAntiCommuting}
\cdbseeprop{Commuting}
\cdbseeprop{ImplicitIndex}
